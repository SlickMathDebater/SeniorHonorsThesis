
\documentclass{amsart}


\newcommand{\tab}{\hspace{4mm}} %Spacing aliases
\newcommand{\shunt}{\vspace{20mm}}

\newcommand{\sd}{\partial} %Squiggle d


\newcommand{\al}{\alpha} %Steal ALL of Dr. Kable's Aliases! MWAHAHAHAHA!
\newcommand{\be}{\beta}
\newcommand{\ga}{\gamma}
\newcommand{\Ga}{\Gamma}
\newcommand{\de}{\delta}
\newcommand{\De}{\Delta}
\newcommand{\ep}{\epsilon}
\newcommand{\vep}{\varepsilon}
\newcommand{\ze}{\zeta}
\newcommand{\et}{\eta}
\newcommand{\tha}{\theta}
\newcommand{\vtha}{\vartheta}
\newcommand{\Tha}{\Theta}
\newcommand{\io}{\iota}
\newcommand{\ka}{\kappa}
\newcommand{\la}{\lambda}
\newcommand{\La}{\Lambda}
\newcommand{\rh}{\rho}
\newcommand{\si}{\sigma}
\newcommand{\Si}{\Sigma}
\newcommand{\ta}{\tau}
\newcommand{\ups}{\upsilon}
\newcommand{\Ups}{\Upsilon}
\newcommand{\ph}{\phi}
\newcommand{\Ph}{\Phi}
\newcommand{\vph}{\varphi}
\newcommand{\vpi}{\varpi}
\newcommand{\ch}{\chi}
\newcommand{\ps}{\psi}
\newcommand{\Ps}{\Psi}
\newcommand{\om}{\omega}
\newcommand{\Om}{\Omega}

\newcommand{\bbA}{\mathbb{A}}
\newcommand{\A}{\mathbb{A}}
\newcommand{\bbB}{\mathbb{B}}
\newcommand{\bbC}{\mathbb{C}}
\newcommand{\C}{\mathbb{C}}
\newcommand{\bbD}{\mathbb{D}}
\newcommand{\bbE}{\mathbb{E}}
\newcommand{\bbF}{\mathbb{F}}
\newcommand{\bbG}{\mathbb{G}}
\newcommand{\G}{\mathbb{G}}
\newcommand{\bbH}{\mathbb{H}}
\newcommand{\HH}{\mathbb{H}}
\newcommand{\bbI}{\mathbb{I}}
\newcommand{\I}{\mathbb{I}}
\newcommand{\bbJ}{\mathbb{J}}
\newcommand{\bbK}{\mathbb{K}}
\newcommand{\bbL}{\mathbb{L}}
\newcommand{\bbM}{\mathbb{M}}
\newcommand{\bbN}{\mathbb{N}}
\newcommand{\N}{\mathbb{N}}
\newcommand{\bbO}{\mathbb{O}}
\newcommand{\bbP}{\mathbb{P}}
\newcommand{\PP}{\mathbb{P}}
\newcommand{\bbQ}{\mathbb{Q}}
\newcommand{\Q}{\mathbb{Q}}
\newcommand{\bbR}{\mathbb{R}}
\newcommand{\R}{\mathbb{R}}
\newcommand{\bbS}{\mathbb{S}}
\newcommand{\bbT}{\mathbb{T}}
\newcommand{\bbU}{\mathbb{U}}
\newcommand{\bbV}{\mathbb{V}}
\newcommand{\bbW}{\mathbb{W}}
\newcommand{\bbX}{\mathbb{X}}
\newcommand{\bbY}{\mathbb{Y}}
\newcommand{\bbZ}{\mathbb{Z}}
\newcommand{\Z}{\mathbb{Z}}

\newcommand{\scrA}{\mathcal{A}}
\newcommand{\scrB}{\mathcal{B}}
\newcommand{\scrC}{\mathcal{C}}
\newcommand{\scrD}{\mathcal{D}}
\newcommand{\scrE}{\mathcal{E}}
\newcommand{\scrF}{\mathcal{F}}
\newcommand{\scrG}{\mathcal{G}}
\newcommand{\scrH}{\mathcal{H}}
\newcommand{\scrI}{\mathcal{I}}
\newcommand{\scrJ}{\mathcal{J}}
\newcommand{\scrK}{\mathcal{K}}
\newcommand{\scrL}{\mathcal{L}}
\newcommand{\scrM}{\mathcal{M}}
\newcommand{\scrN}{\mathcal{N}}
\newcommand{\scrO}{\mathcal{O}}
\newcommand{\scrP}{\mathcal{P}}
\newcommand{\scrQ}{\mathcal{Q}}
\newcommand{\scrR}{\mathcal{R}}
\newcommand{\scrS}{\mathcal{S}}
\newcommand{\scrT}{\mathcal{T}}
\newcommand{\scrU}{\mathcal{U}}
\newcommand{\scrV}{\mathcal{V}}
\newcommand{\scrW}{\mathcal{W}}
\newcommand{\scrX}{\mathcal{X}}
\newcommand{\scrY}{\mathcal{Y}}
\newcommand{\scrZ}{\mathcal{Z}}

\usepackage{fancyhdr}
\usepackage{amssymb}

\usepackage{amsmath}
\usepackage{slashed}
\usepackage[mathscr]{euscript}

\newtheorem{thm}{Theorem}[section]
\newtheorem{prop}[thm]{Proposition}
\newtheorem{lem}[thm]{Lemma}
\newtheorem{cor}[thm]{Corollary} 

\theoremstyle{definition}
\newtheorem{definition}[thm]{Definition}
\newtheorem{example}[thm]{Example}



\begin{document}

\title{The Existence of a Continuous Function Whose Fourier Series Diverges at a Point}

\maketitle

\begin{abstract}
This paper reconstructs a proof of the existence of a continuous function whose fourier series diverges at a point.
\end{abstract}

\section{Notation}

\begin{definition}If $X$ is a metric space, $r$ is a real number with $r>0$, and $x \in X$, then we say that $X_r(x)$ is the $r$-ball centered at $x$.

\end{definition}

\section{Actual Work}

\subsection{Baire's Theorem}

Exposition about how this is the driving force for the entire proof goes here.

\begin{thm} Baire's Theorem: If $X$ is a complete metric space, and $E_1, E_2, \ldots$ are dense, open subsets of $X$, then

\begin{displaymath}
\bigcap\limits_{n = 1}^{\infty} E_n
\end{displaymath}

is dense.
\end{thm}

\begin{proof}
Let $X$ be a complete metric space and $E_1, E_2, \ldots$ be dense open subsets of $X$.

We proceed by showing that $\bigcap\limits_{n = 1}^{\infty} E_n$ intersects every open set.

Let $U$ be an open set.

Define $B_1 = U$.

$E_1 \cap B_1 \neq \emptyset$, because $E_1$ is dense.

So there is an $x_1 \in E_1 \cap B_1$.

$E_1 \cap B_1$ is open, because $E_1$ and $B_1$ are both open.

So there is an $r_1 >0$ such that $X_{r_1}(x_1) \subset E_1 \cap B_1$.

Define $\ep_1 = $min$(1, r_1)$.

Define $B_2 = X_{\ep_1}(x_1)$, and define $B_2'= X_{r_1}(x_1)$.

$E_2 \cap B_2 \neq \emptyset$, because $E_2$ is dense.

So there is an $x_2 \in E_2 \cap B_2$.

$E_2 \cap B_2$ is open, because $E_2$ and $B_2$ are both open.

So there is an $r_2 >0$ such that $X_{r_2}(x_2) \subset E_2 \cap B_2$.

Define $\ep_2 = $min$(r_2, \frac{r_1}{2})$.

Define a sequence of open sets $(B_n)$ and $(B_n')$, as well as points $(x_n)$ and real numbers $(r_n)$. "similarly". (How do I make this sound right?)

Now, the sequence $(x_n)$ is cauchy, by construction.

So, because $X$ is complete, this means that $(x_n) \to x$ for some $x \in X$.

Observe that $x_n \in B_n$ for all $n \in \N$.

Also observe that $B_{n+1} \subset B_n$ for all $n \in \N$.

So $x_n \in \bigcap\limits_{i =1}^n$ for all $n \in \N$.

This means that $x \in \overline{B_n}$ for all $n \in \N$.

Now, $\overline{B_{n+1}} \subset E_n$ for all $n \in \N$:

\tab $E_n \supset E_n \cap B_n$.

\tab $E_n \cap B_n \supset B_{n+1}'$, by construction.

\tab $B_{n+1}' \supset \overline{B_{n+1}}$, because $B_{n+1}'=X_{r_n}(x_n)$ and $B_{n+1} \subset X_{\frac{r_n}{2}}(x_n)$. (This is not clear but I don't know how to make this clear.) 

This means that $x \in E_n$ for all $n \in \N$.

So $x \in \bigcap\limits_{n = 1}^{\infty} E_n$.

Also, $x \in U$.

So $x \in U \cap \bigcap\limits_{n = 1}^{\infty} E_n$

So $U \cap \bigcap\limits_{n = 1}^{\infty} E_n \neq \emptyset$.

So if $U$ is an open set, then $U \cap \bigcap\limits_{n = 1}^{\infty} E_n \neq \emptyset$.

In other words, $\bigcap\limits_{n = 1}^{\infty} E_n$ is dense.
\end{proof}

\begin{thm}
There exists a continuous function whose fourier series diverges at a point.
\end{thm}

\begin{proof}

Proof goes here.

\end{proof}

\end{document}