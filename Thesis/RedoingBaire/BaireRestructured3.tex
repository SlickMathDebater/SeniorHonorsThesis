
\documentclass{amsart}


\newcommand{\tab}{\hspace{4mm}} %Spacing aliases
\newcommand{\shunt}{\vspace{20mm}}

\newcommand{\sd}{\partial} %Squiggle d


\newcommand{\al}{\alpha} %Steal ALL of Dr. Kable's Aliases! MWAHAHAHAHA!
\newcommand{\be}{\beta}
\newcommand{\ga}{\gamma}
\newcommand{\Ga}{\Gamma}
\newcommand{\de}{\delta}
\newcommand{\De}{\Delta}
\newcommand{\ep}{\epsilon}
\newcommand{\vep}{\varepsilon}
\newcommand{\ze}{\zeta}
\newcommand{\et}{\eta}
\newcommand{\tha}{\theta}
\newcommand{\vtha}{\vartheta}
\newcommand{\Tha}{\Theta}
\newcommand{\io}{\iota}
\newcommand{\ka}{\kappa}
\newcommand{\la}{\lambda}
\newcommand{\La}{\Lambda}
\newcommand{\rh}{\rho}
\newcommand{\si}{\sigma}
\newcommand{\Si}{\Sigma}
\newcommand{\ta}{\tau}
\newcommand{\ups}{\upsilon}
\newcommand{\Ups}{\Upsilon}
\newcommand{\ph}{\phi}
\newcommand{\Ph}{\Phi}
\newcommand{\vph}{\varphi}
\newcommand{\vpi}{\varpi}
\newcommand{\ch}{\chi}
\newcommand{\ps}{\psi}
\newcommand{\Ps}{\Psi}
\newcommand{\om}{\omega}
\newcommand{\Om}{\Omega}

\newcommand{\bbA}{\mathbb{A}}
\newcommand{\A}{\mathbb{A}}
\newcommand{\bbB}{\mathbb{B}}
\newcommand{\bbC}{\mathbb{C}}
\newcommand{\C}{\mathbb{C}}
\newcommand{\bbD}{\mathbb{D}}
\newcommand{\bbE}{\mathbb{E}}
\newcommand{\bbF}{\mathbb{F}}
\newcommand{\bbG}{\mathbb{G}}
\newcommand{\G}{\mathbb{G}}
\newcommand{\bbH}{\mathbb{H}}
\newcommand{\HH}{\mathbb{H}}
\newcommand{\bbI}{\mathbb{I}}
\newcommand{\I}{\mathbb{I}}
\newcommand{\bbJ}{\mathbb{J}}
\newcommand{\bbK}{\mathbb{K}}
\newcommand{\bbL}{\mathbb{L}}
\newcommand{\bbM}{\mathbb{M}}
\newcommand{\bbN}{\mathbb{N}}
\newcommand{\N}{\mathbb{N}}
\newcommand{\bbO}{\mathbb{O}}
\newcommand{\bbP}{\mathbb{P}}
\newcommand{\PP}{\mathbb{P}}
\newcommand{\bbQ}{\mathbb{Q}}
\newcommand{\Q}{\mathbb{Q}}
\newcommand{\bbR}{\mathbb{R}}
\newcommand{\R}{\mathbb{R}}
\newcommand{\bbS}{\mathbb{S}}
\newcommand{\bbT}{\mathbb{T}}
\newcommand{\bbU}{\mathbb{U}}
\newcommand{\bbV}{\mathbb{V}}
\newcommand{\bbW}{\mathbb{W}}
\newcommand{\bbX}{\mathbb{X}}
\newcommand{\bbY}{\mathbb{Y}}
\newcommand{\bbZ}{\mathbb{Z}}
\newcommand{\Z}{\mathbb{Z}}

\newcommand{\scrA}{\mathcal{A}}
\newcommand{\scrB}{\mathcal{B}}
\newcommand{\scrC}{\mathcal{C}}
\newcommand{\scrD}{\mathcal{D}}
\newcommand{\scrE}{\mathcal{E}}
\newcommand{\scrF}{\mathcal{F}}
\newcommand{\scrG}{\mathcal{G}}
\newcommand{\scrH}{\mathcal{H}}
\newcommand{\scrI}{\mathcal{I}}
\newcommand{\scrJ}{\mathcal{J}}
\newcommand{\scrK}{\mathcal{K}}
\newcommand{\scrL}{\mathcal{L}}
\newcommand{\scrM}{\mathcal{M}}
\newcommand{\scrN}{\mathcal{N}}
\newcommand{\scrO}{\mathcal{O}}
\newcommand{\scrP}{\mathcal{P}}
\newcommand{\scrQ}{\mathcal{Q}}
\newcommand{\scrR}{\mathcal{R}}
\newcommand{\scrS}{\mathcal{S}}
\newcommand{\scrT}{\mathcal{T}}
\newcommand{\scrU}{\mathcal{U}}
\newcommand{\scrV}{\mathcal{V}}
\newcommand{\scrW}{\mathcal{W}}
\newcommand{\scrX}{\mathcal{X}}
\newcommand{\scrY}{\mathcal{Y}}
\newcommand{\scrZ}{\mathcal{Z}}

\usepackage{fancyhdr}
\usepackage{amssymb}

\usepackage{amsmath}
\usepackage{slashed}
\usepackage[mathscr]{euscript}

\newtheorem{thm}{Theorem}[section]
\newtheorem{prop}[thm]{Proposition}
\newtheorem{lem}[thm]{Lemma}
\newtheorem{cor}[thm]{Corollary} 

\theoremstyle{definition}
\newtheorem{definition}[thm]{Definition}
\newtheorem{example}[thm]{Example}



\begin{document}

\title{The Existence of a Continuous Function Whose Fourier Series Diverges at a Point}
\author{Raymond Maxwell Jeter}

\maketitle

\begin{abstract}
There exists a continuous function whose Fourier series diverges at a point.
\end{abstract}

\section{Notation}

\begin{definition}If $X$ is a metric space, $r$ is a real number with $r>0$, and $x \in X$, then $B(x,r)$ is the $r$-ball centered at $x$.

\end{definition}

\section{Actual Work}

\subsection{Baire's Theorem}

Exposition about how this is the driving force for the entire proof goes here.

In order to facilitate the proof of Baire's Theorem, we prove the following lemma:

\begin{lem}
If $X$ is a complete metric space, $C_1, C_2 \ldots$ are closed, nonempty subsets of $X$, $C_{n+1} \subset C_n$ for all $n \in \N$ and $diam(C_n) \to 0$, then $\bigcap C_n$ has exactly one point.
\end{lem}

\begin{proof}
First we show that $\bigcap C_n$ is nonempty: 

Observe that $\bigcap\limits_{i=0}^n C_i$ is nonempty for all $n \in \N$. 
So there is a sequence $(x_n)$ such that $x_n \in \bigcap\limits_{i=0}^n C_i$ for all $n \in \N$. 
The sequence $(x_n)$ is Cauchy because $diam(C_n) \to 0$, so $(x_n) \to x$ for some $x \in X$.
Now, $x \in \bigcap\limits_{i=0}^n C_i$ for all $n \in \N$.
So $x \in \bigcap C_n$: $\bigcap C_n \neq \emptyset$.

Next, we show that $\bigcap C_n$ has exactly one point:

Let $x, y \in \bigcap C_n$, with $x \neq y$. 
Then $d(x,y) = \ep$ for some $\ep >0$. 
However, $x,y \in C_n$ for all $n \in \N$, so for all $n \in \N$ there is a pair of points $x,y$ such that $d(x,y) = \ep$; 
so $diam(C_n) \geq \ep$ for all $n \in \N$. 
So $diam(C_n) \slashed{\to} 0$, contrary to our assumption.
\end{proof}

\begin{thm}If $X$ is a complete metric space, and $E_1, E_2, \ldots$ are dense, open subsets of $X$, then

\begin{displaymath}
\bigcap\limits_{n = 1}^{\infty} E_n
\end{displaymath}

is dense.
\end{thm}

\begin{proof}
Let $X$ be a complete metric space and $E_1, E_2, \ldots$ be dense open subsets of $X$. Let $U$ be a nonempty open subset of $X$.

For each $n \in \N$ we have $E_n \cap U$ nonempty and open: there is a point, $x_n \in E_n \cap U$, with a ball, $B(x_n, r_n)$, such that$B(x_n, r_n) \subset E_n \cap U$. 
So for each $n \in \N$ we can choose $\ep_n$ such that $\ep_n < r_n$ and $(\ep_n) \to 0$. %Is this how we're supposed to do it?
So we have $\overline{B(x_n,\ep_n)} \subset E_n \cap U$ for each $n \in \N$.

Define $C_1 = \overline{B(x_1,\ep_1)}$ and $C_n = \overline{B(x_1,\ep_1)} \cap \bigcap\limits_{i=1}^n C_i$.
For all $n \in \N$,  $C_n$ is closed, $C_{n+1} \subset C_n$, $diam(C_n) \to 0$, and $C_n \neq \emptyset$.
So $\bigcap C_n$ has exactly one point by the above lemma. %I should learn how to reference lemmas

Also, for all $n \in \N$, $C_n \subset E_n \cap U$.
So $\bigcap C_n \subset \bigcap E_n \cap U$.
So $\bigcap E_n \cap U$ has at least one point; it is nonempty.

So $\bigcap E_n$ intersects every nonempty open set, $\bigcap E_n$ is dense.

\end{proof}

\begin{thm}
There exists a continuous function whose fourier series diverges at a point.
\end{thm}

\begin{proof}

Proof goes here.

\end{proof}

\end{document}