%Comments go here.
%Remember, Max: No multiline comments here. LaTeX land isn't quite as good as it should be.
%Anyway, let's get on this:
%Doing numbering like this from now on:
%0.abc: a: math draft #. b: exposition draft # c: formatting draft number.

%Current status:
%Barebones: ALL CLEAR. 

%Exposition:
%Introduction: Not researched.
%Proof Outline: Planned. Not researched.
%Outro: Researched. Written. Not approved.

%Formatting:
%Will figure out later.

%Citations:
%Must figure out almost immediately.


\documentclass{amsart}


\newcommand{\tab}{\hspace{4mm}} %Spacing aliases
\newcommand{\shunt}{\vspace{20mm}}

\newcommand{\sd}{\partial} %Squiggle d


\newcommand{\al}{\alpha} %Steal ALL of Dr. Kable's Aliases! MWAHAHAHAHA!
\newcommand{\be}{\beta}
\newcommand{\ga}{\gamma}
\newcommand{\Ga}{\Gamma}
\newcommand{\de}{\delta}
\newcommand{\De}{\Delta}
\newcommand{\ep}{\epsilon}
\newcommand{\vep}{\varepsilon}
\newcommand{\ze}{\zeta}
\newcommand{\et}{\eta}
\newcommand{\tha}{\theta}
\newcommand{\vtha}{\vartheta}
\newcommand{\Tha}{\Theta}
\newcommand{\io}{\iota}
\newcommand{\ka}{\kappa}
\newcommand{\la}{\lambda}
\newcommand{\La}{\Lambda}
\newcommand{\rh}{\rho}
\newcommand{\si}{\sigma}
\newcommand{\Si}{\Sigma}
\newcommand{\ta}{\tau}
\newcommand{\ups}{\upsilon}
\newcommand{\Ups}{\Upsilon}
\newcommand{\ph}{\phi}
\newcommand{\Ph}{\Phi}
\newcommand{\vph}{\varphi}
\newcommand{\vpi}{\varpi}
\newcommand{\ch}{\chi}
\newcommand{\ps}{\psi}
\newcommand{\Ps}{\Psi}
\newcommand{\om}{\omega}
\newcommand{\Om}{\Omega}

\newcommand{\bbA}{\mathbb{A}}
\newcommand{\A}{\mathbb{A}}
\newcommand{\bbB}{\mathbb{B}}
\newcommand{\bbC}{\mathbb{C}}
\newcommand{\C}{\mathbb{C}}
\newcommand{\bbD}{\mathbb{D}}
\newcommand{\bbE}{\mathbb{E}}
\newcommand{\bbF}{\mathbb{F}}
\newcommand{\bbG}{\mathbb{G}}
\newcommand{\G}{\mathbb{G}}
\newcommand{\bbH}{\mathbb{H}}
\newcommand{\HH}{\mathbb{H}}
\newcommand{\bbI}{\mathbb{I}}
\newcommand{\I}{\mathbb{I}}
\newcommand{\bbJ}{\mathbb{J}}
\newcommand{\bbK}{\mathbb{K}}
\newcommand{\bbL}{\mathbb{L}}
\newcommand{\bbM}{\mathbb{M}}
\newcommand{\bbN}{\mathbb{N}}
\newcommand{\N}{\mathbb{N}}
\newcommand{\bbO}{\mathbb{O}}
\newcommand{\bbP}{\mathbb{P}}
\newcommand{\PP}{\mathbb{P}}
\newcommand{\bbQ}{\mathbb{Q}}
\newcommand{\Q}{\mathbb{Q}}
\newcommand{\bbR}{\mathbb{R}}
\newcommand{\R}{\mathbb{R}}
\newcommand{\bbS}{\mathbb{S}}
\newcommand{\bbT}{\mathbb{T}}
\newcommand{\bbU}{\mathbb{U}}
\newcommand{\bbV}{\mathbb{V}}
\newcommand{\bbW}{\mathbb{W}}
\newcommand{\bbX}{\mathbb{X}}
\newcommand{\bbY}{\mathbb{Y}}
\newcommand{\bbZ}{\mathbb{Z}}
\newcommand{\Z}{\mathbb{Z}}

\newcommand{\scrA}{\mathcal{A}}
\newcommand{\scrB}{\mathcal{B}}
\newcommand{\scrC}{\mathcal{C}}
\newcommand{\scrD}{\mathcal{D}}
\newcommand{\scrE}{\mathcal{E}}
\newcommand{\scrF}{\mathcal{F}}
\newcommand{\scrG}{\mathcal{G}}
\newcommand{\scrH}{\mathcal{H}}
\newcommand{\scrI}{\mathcal{I}}
\newcommand{\scrJ}{\mathcal{J}}
\newcommand{\scrK}{\mathcal{K}}
\newcommand{\scrL}{\mathcal{L}}
\newcommand{\scrM}{\mathcal{M}}
\newcommand{\scrN}{\mathcal{N}}
\newcommand{\scrO}{\mathcal{O}}
\newcommand{\scrP}{\mathcal{P}}
\newcommand{\scrQ}{\mathcal{Q}}
\newcommand{\scrR}{\mathcal{R}}
\newcommand{\scrS}{\mathcal{S}}
\newcommand{\scrT}{\mathcal{T}}
\newcommand{\scrU}{\mathcal{U}}
\newcommand{\scrV}{\mathcal{V}}
\newcommand{\scrW}{\mathcal{W}}
\newcommand{\scrX}{\mathcal{X}}
\newcommand{\scrY}{\mathcal{Y}}
\newcommand{\scrZ}{\mathcal{Z}}

\newcommand{\diam}{\text{\rm diam}}
\newcommand{\Lip}{\text{\rm Lip}}

\newcommand{\colorcomment}[2]{\textcolor{#1}{#2}} %First of these leaves in comments. Second one kills them.
%\newcommand{\colorcomment}[2]{}

\newcommand{\absval}[1]{\left| #1 \right|}
\newcommand{\norm}[1]{\|#1\|}

\usepackage{fancyhdr}
\usepackage{amssymb}
\usepackage[usenames,dvipsnames]{color}
\usepackage{amsmath}
\usepackage{verbatim} 
\usepackage{slashed}
\usepackage[mathscr]{euscript}
\usepackage{graphicx}

\newtheorem{thm}{Theorem}[section]
\newtheorem{prop}[thm]{Proposition}
\newtheorem{lem}[thm]{Lemma}
\newtheorem{cor}[thm]{Corollary} 

\theoremstyle{definition}
\newtheorem{definition}[thm]{Definition}
\newtheorem{example}[thm]{Example}



\begin{document}

\title{The Existence of a Continuous Function Whose Fourier Series Diverges at a Point}
\author{Raymond Maxwell Jeter}

\begin{abstract}
There exists a continuous function whose Fourier series diverges at a point. 
In fact, the set of continuous functions whose Fourier series diverges at $0$ is dense in $C(T)$.
\end{abstract}

\maketitle

\section{Introduction}

In The Analytical Theory of Heat, Fourier claimed that any periodic function can be expressed as a Fourier Series, even if it was discontinuous. \colorcomment{red}{(cite it)}
Indeed, many great mathematicians of the time, such as Cauchy, had produced flawed proofs of this. 
After having proved a result that was ``really close'' to what was desired, Dirichlet seemed convinced of the truth of this. 
Over time, other great mathematicians, such as Dedekind, Riemann, and Weierstrass were also convinced. 
However, Du Bois-Reymond constructed a counterexample! \colorcomment{red}{(cite Korner)}

This paper is written as a nod to the rich history of this problem.

\section{Notation}

\begin{definition}
We say that $T = \R / 2\pi \Z$.
\end{definition}

\begin{definition}
If $X$ is a metric space, $r$ is a real number with $r>0$, and $x \in X$, then $B(x,r)$ is the $r$-ball centered at $x$.
\end{definition}

\begin{definition}
If $f \in C(T)$, $n \in \N$, and $x \in T$, then the $n$th partial sum of the Fourier series of $f$ at $x$ is 

\begin{displaymath}
S_n(f,x) = \sum\limits_{j=-n}^n c_j e^{ijx},
\end{displaymath}
where $c_j = \frac{1}{2 \pi} \int\limits_0^{2\pi} f(t)e^{-ijt} dt$.
\end{definition}

\begin{definition}
If $f$ is a bounded function on $T$, then $\norm{f} = \sup\limits_{t \in T} \{\absval{f(t)}\}$
\end{definition}

\begin{definition}
For each $n \in \N$, we define $D_n: T \to \R$ by
\begin{displaymath}
   D_n(x) = \sum\limits_{j=-n}^n e^{ijx}
\end{displaymath} 
\end{definition}

Also, we freely use the following well-known theorems:

\begin{thm}
For all $n \in \N$, 
\begin{displaymath}
   D_n(x) = \left\{
     \begin{array}{lr}
       \frac{\sin((n+1/2)x)}{\sin(x/2)}, &(x \neq 0),\\
       1+2n, &(x=0).
     \end{array}
   \right.
\end{displaymath}
\end{thm}

\begin{thm}
For any $f \in C(T)$, $(D_n \ast f)(x) = S_n(f,x)$. 
\end{thm}

\begin{thm}
For any $f \in C(T)$, $f$ is bounded.
\end{thm}

\section{Continuous Functions With Some Partial Sum of Their Fourier Series Larger Than A Given Real Number}

A rough sketch of the proof follows. 
First, for each $A \in \R$ we build a continuous function, $f$, with $\norm{f} = 1$ and with some partial sum of its Fourier series larger than $A$ at $0$. 
We then consider sets of continuous functions with some partial sum of their Fourier series larger than $n$ for each $n \in \N$.
We show that these sets are open and dense. 
After that, we apply Baire's Theorem, which gives us a dense, open set of functions whose Fourier series are unbounded at the point $0$. 
This dense, open set is nonempty; we have a continuous function whose Fourier series diverges at a point.

Those who are familiar with the subject area will recognize that this is similar to a standard proof of the existence of a continuous function whose Fourier series diverges. 
In fact, this proof is based on an observation by Dr. David C. Ullrich about a proof that involves functional analysis. 
Dr. Ullrich saw that the parts of the proof involving functional analysis can be stripped away. 
He then produced an outline of a proof that is much more accessible to a student who is just beginning to study Fourier analysis.

\subsection{The Construction of Bounded, Continuous Functions With Some Partial Sum of Their Fourier Series Larger Than A Given Real Number}\hspace*{\fill}

Inspiration for constructing the titular functions comes from the proof of the following theorem.

\begin{thm}
For each $n \in \N$, define $c_n \in \R$ to be

\begin{displaymath}
c_n = \frac{1}{2 \pi} \int\limits_0^{2\pi} \absval{D_n(x)} dx.
\end{displaymath}

The sequence $c_n \to \infty$.
\end{thm}

\begin{proof}

First, $\absval{D_n(x)} \geq \frac{1/2}{\sin(x/2)}$ if $x \in [\frac{\pi k}{6(n+1/2)},\frac{5 \pi k}{6(n+1/2)}]$ for some $k \in \N$.

Recall that for all $x \in [0, 2\pi]$, $x > \absval{\sin(x)}$. 
So, $\absval{\frac{2}{x}} < \absval{\frac{1}{\sin(x/2)}}$ for all $x \in (0,2\pi)$.

This means that $\absval{D_n(x)} \geq \frac{1}{x}$ on those intervals, which naturally produces the inequality $\int\limits_0^{2\pi} \absval{D_n(x)} dx \geq \sum\limits_{j=1}^n \frac{2 \pi}{6n \frac{5\pi j}{6(n+1/2)}}$. 

This means that 

\begin{align*}
c_n &\geq \frac{1}{2 \pi} \sum\limits_{j=1}^n \frac{2 \pi}{6n \frac{5\pi j}{6(n+1/2)}}\\
&= \sum\limits_{j=1}^n \frac{1}{6n \frac{5\pi j}{6(n+1/2)}}\\
&= \sum\limits_{j=1}^n \frac{(n+1/2)/{n}}{5\pi j}\\
&\geq \sum\limits_{j=1}^n \frac{1}{5\pi j}
\end{align*}

So $c_n \geq \frac{1}{5 \pi} \sum\limits_{j=1}^n \frac{1}{ j}$, and $\sum\limits_{j=1}^n \frac{1}{j} \to \infty$. 
Thus, $c_n \to \infty$.

\end{proof}

It is important to determine how large the partial sum of a function's Fourier series can get at $0$. 
So we make the following definition:

\begin{definition}
If $f \in C(T)$, then $Mf = \sup\{\absval{S_n(f,0)} : n \in \N\}$ 
\end{definition}

Ultimately, we want to build a function with $Mf = \infty$.
One way to do this is to build a sequence of uniformly bounded functions, $f_n$, such that $Mf_n \to \infty$; 
if we can do that, then we should be able to build the function we want.
We build this sequence rather naturally out of the proof of the above theorem.

\begin{thm}
For each $A \in \R$, there is an $f \in C(T)$ such that $\norm{f} = 1$ and $Mf > A$. 
\end{thm}

\begin{proof}
For each $n \in \N$, choose $g_n: T \to \R$ with $\norm{g_n} = 1$, $g_nD_n(t) \geq 0$ for all $t \in T$,  $g_n(t) = 1$ if $t \in [\frac{\pi k}{6(n+1/2)},\frac{5 \pi k}{6(n+1/2)}]$ for some odd $k \in \N$  and $g_n(t) = -1$ if $t \in [\frac{\pi k}{6(n+1/2)},\frac{5 \pi k}{6(n+1/2)}]$ for some even $k \in \N$.

So, similar to the above,

\begin{align*}
g_n \ast D_n(0) &\geq \frac{1}{2 \pi} \sum\limits_{j=1}^n \frac{2 \pi}{6n \frac{5\pi j}{6(n+1/2)}}\\
&= \sum\limits_{j=1}^n \frac{1}{6n \frac{5\pi j}{6(n+1/2)}}\\
&= \sum\limits_{j=1}^n \frac{(n+1/2)/{n}}{5\pi j}\\
&\geq \sum\limits_{j=1}^n \frac{1}{5\pi j}
\end{align*}

So $g_n \ast D_n(0) \to \infty$. In other words, for each $A$ there is an $n \in \N$ such that $Mg_n > A$, which gives us what we wanted.

\end{proof}

\begin{thm}
For each $\vep >0$ and each $A \in \R$, there is an $f \in C(T)$ such that $\norm{f} = \vep $ and $Mf > A$. 

\end{thm}

\begin{proof}
There is an $f \in C(T)$ such that $\norm{f} = 1$ and $Mf > \frac{A}{\vep}$.

Observe that $\norm{\vep f} = \vep$ and $M(\vep f) > A$.

\end{proof}

\subsection{Building Open, Dense Sets of Continuous Functions With Some Partial Sum of its Fourier Series Larger Than A Given Real Number} \hspace*{\fill}\\

Next, we consider the following collection of sets: for each $n \in \N$, we define

\begin{displaymath}
E_n = \{f \in C(T) : Mf > n\}
\end{displaymath}

\begin{thm}
For each $n \in \N$, $E_n$ is dense.
\end{thm}

\begin{proof}

We show that given $\vep>0$ and $f \in C(T)$, $B(f,\vep) \cap E_n$ is nonempty. 

If $Mf > n$, then $f \in E_n$ and $f \in B(f, \vep)$, so $B(f, \vep) \cap E_n$ is nonempty.

Next, if $Mf \leq n$, there is a $g$ such that $Mg > 2n$ and $\norm{g} < \vep$. 
Now, $M(f+g) \geq n$, so $f+g \in E_n$. 
Also, $\norm{f+g -f} < \vep$, so that $f+g \in B(f, \vep)$. 
So $B(f, \vep) \cap E_n$ is nonempty.

\end{proof}

\begin{thm}
For each $n \in \N$, $E_n$ is open.
\end{thm}

\begin{proof}
Consider $\la_n: C(T) \to \R$ given by

\begin{displaymath}
\la_n(f) = \absval{S_n(f,0)}
\end{displaymath}

Now, consider the following:

\begin{align*}
E_n &= \{f \in C(T): Mf> n\} \\
&= \bigcup\limits_{N \in \N} \{f \in C(T) : \la_N (f)> n\}\\
&= \bigcup\limits_{N \in \N} \la_N^{-1}((n, \infty))
\end{align*}

It is sufficient to show that $\la_n$ is continuous for all $n \in \N$; if so, then the right hand side of the above equation is open.
We do this by showing that $\la_n$ is Lipschitz.

If $f, g \in C(T)$, then we have:

\begin{align*}
\absval{\la_n(f) - \la_n(g)} &= \absval{\absval{f \ast D_n(0)} - \absval{g \ast D_n(0)}} \\
&\leq \absval{f \ast D_n(0) - g \ast D_n(0)} \\
&= \absval{(f-g) \ast D_n(0)} \\
&= \absval{\int\limits_0^{2\pi} (f(t)-g(t))D_n(t)dt} \\
&\leq \int\limits_0^{2\pi} \absval{(f(t)-g(t))D_n(t)}dt \\
&= \int\limits_0^{2\pi} \absval{(f(t)-g(t))}\absval{D_n(t)}dt \\
&\leq \int\limits_0^{2\pi} \norm{f-g}\absval{D_n(t)}dt \\
&= \norm{f-g} \int\limits_0^{2\pi} \absval{D_n(t)}dt
\end{align*}

This means that for any $f,g \in C(T)$, $\absval{\la_n(f) - \la_n(g)} \leq c_n \norm{f-g}$ for some constant $c_n$; thus, $\la_n$ is Lipschitz.

\end{proof}

\begin{thm}
The metric space $C(T)$ is complete.
\end{thm}

\begin{proof}
We first show that the set of bounded functions, $B(T)$, is complete.
So, we want to show that any Cauchy sequence of bounded functions, $(f_n)$, converges to a bounded function.

Note that $\R$ is complete: so for each $x \in T$, $(f_n(x))$ converges to a point. We define $f: T \to \R$ by $f(x) = \lim\limits_{n \to \infty} f_n(x)$.

Now, $f$ is bounded: 
There is an $N$ such that for all $n,m \geq N$, $\norm{f_n-f_m} < 1$.
If $m \geq N$ and $x \in X$, then $\absval{f_N(x) - f_m(x)} < 1$.
So by taking a limit as $m \to \infty$, we have $\absval{f_N(x) - f(x)} \leq 1$ for all $x \in X$.
Because $f_N$ is bounded, this means that $f$ is bounded.

Now: let $\vep >0$. Choose $N$ such that for all $n,m \geq N$, $\norm{f_n-f_m} < \frac{\vep}{2}$. 
Then for all $x \in T$ and $n,m \geq N$, we have $\absval{f_n(x) - f_m(x)} < \frac{\vep}{2}$.
By taking a limit as $m \to \infty$, we have for all $x \in T$ and $n \geq N$, $\absval{f_n(x) - f(x)} \leq \frac{\vep}{2}$.

So $\norm{f_n - f} \leq \frac{\vep}{2} < \vep$ for all $n \geq N$; in other words, $f_n \to f$.

Next, recall that $C(T)$ is a closed subset of $B(T)$; because $C(T)$ is a closed subset of a complete metric space, it is complete.

\end{proof}

\subsection{Baire's Theorem} \hspace*{\fill}

We now have a collection of open, dense sets in a complete metric space. 
We want to show that their intersection is nonempty; a dramatic way of doing this is to show that the intersection is dense. 
A flair for the dramatic is necessary in life and mathematics, so this is how we should proceed.

In order to facilitate the proof of Baire's Theorem, we prove the following lemma:

\begin{lem}
If $X$ is a complete metric space, $C_1, C_2 \ldots$ are closed, nonempty subsets of $X$, $C_{n+1} \subset C_n$ for all $n \in \N$ and $\diam(C_n) \to 0$, then $\bigcap C_n$ has exactly one point.
\end{lem}

\begin{proof}
First we show that $\bigcap C_n$ is nonempty: 

Observe that $\bigcap\limits_{i=0}^n C_i$ is nonempty for all $n \in \N$. 
So there is a sequence $(x_n)$ such that $x_n \in \bigcap\limits_{i=0}^n C_i$ for all $n \in \N$. 
The sequence $(x_n)$ is Cauchy because $\diam(C_n) \to 0$, so $(x_n) \to x$ for some $x \in X$.
Now, $x \in \bigcap\limits_{i=0}^n C_i$ for all $n \in \N$.
So $x \in \bigcap C_n$: $\bigcap C_n \neq \emptyset$.

Next, we show that $\bigcap C_n$ has exactly one point:

Let $x, y \in \bigcap C_n$, with $x \neq y$. 
Then $d(x,y) = \vep$ for some $\vep >0$. 
However, $x,y \in C_n$ for all $n \in \N$,
so $\diam(C_n) \geq \vep$ for all $n \in \N$. 
So $\diam(C_n) \slashed{\to} 0$, contrary to our assumption.
\end{proof}

\begin{thm}[Baire's Theorem]If $X$ is a complete metric space, and $E_1, E_2, \ldots$ are dense, open subsets of $X$, then
\begin{displaymath}
\bigcap\limits_{n = 1}^{\infty} E_n
\end{displaymath}
is dense.
\end{thm}

\begin{proof}
We need to show that if $U$ is a nonempty open set, then $\bigcap E_n \cap U$ is nonempty.

First, $E_1 \cap U$ is nonempty and open.
So there is a ball, $B(x_1,r_1) \subset E_1 \cap U$.
By taking $\vep_1 = \min(\frac{r_1}{2},1)$, we have $\overline{B(x_1,\vep_1)} \subset E_1 \cap U$.
Define $C_1 = \overline{B(x_1,\vep_1)}$.
Observe that $C_1$ is closed and nonempty. Also, $C_1 \subset U$.

Now, assume that for each $n \in \N$ such that $n < N$, we have constructed $C_{n}$, $x_{n}$, and $\vep_{n}$ such that $C_{n} = \overline{B(x_{n},\vep_{n})}$, $C_n \subset C_{n-1}$ (if $n >1$), $C_{n} \subset E_{n}$, and $\vep_{n} < \frac{1}{n}$.

Note that $E_N \cap B(x_{N-1},\vep_{N-1})$ is nonempty and open.
So there is a ball, $B(x_N,r_N) \subset E_N \cap B(x_{N-1},\vep_{N-1})$.
By taking $\vep_N = \min(\frac{r_N}{2},\frac{1}{N})$, we have $\overline{B(x_N,\vep_N)} \subset E_N \cap B(x_{N-1},\vep_{N-1})$.
Define $C_N = \overline{B(x_N,\vep_N)}$.
Observe that $C_N$ is closed and nonempty. Also, $C_N \subset C_{N-1}$.

So we have a collection of closed, nonempty, nested sets.
Also, note that $\diam(C_n) \to 0$ because $\vep_n \to 0$. 
So $\bigcap\limits_{n=1}^\infty C_n$ has exactly one point.
This point is also in $U$, because $C_1 \subset U$.
Also, this point is in $E_n$ for all $n \in \N$, because $C_n \subset E_n$ for all $n \in \N$.
So $\bigcap\limits_{n=1}^\infty E_n \cap U$ is nonempty.

\end{proof}

\section{The Existence of a Continuous Function Whose Fourier Series Diverges at a Point}

\begin{thm}
There exists a continuous function whose Fourier series diverges at a point.
In fact, the set of functions in $C(T)$ whose Fourier series diverges at $0$ is dense.
\end{thm}

\begin{proof}

Recall that 
$E_n = \{f \in C(T) : Mf > n\}$
is a collection of open, dense subsets of a complete space, $C(T)$. 
So Baire's Theorem applies: $\bigcap\limits_{n=1}^\infty E_n $ is dense.
This is the set of functions, $f \in C(T)$, such that $Mf >n$ for all $n \in \N$.
This is a subset of the set of functions whose Fourier series diverges at $0$.
So the set of functions whose Fourier series diverges at $0$ contains a dense subset, it is dense.

Dense subsets of $C(T)$ are nonempty, so there is a continuous function whose Fourier series diverges at a point.

\end{proof}

\section{Closing Remarks}

Recall Dini's Test:

\begin{thm}[Dini's Test]
If $f \in C(T)$, $x_0 \in T$, and $\int\limits_{-\de}^{\de} \absval{\frac{f(x+x_0) - f(x_0)}{x}} dx < \infty$ for some $\de >0$, then the Fourier series of $f$ converges at $x_0$.
\end{thm}

The condition required to apply Dini's test is ``just a little better than continuity''; if $f$ is continuous, then the term $f(x+x_0) - f(x_0)$ approaches $0$ as $x$ approaches $0$.

Consider the function we have shown exists: it must fail the hypotheses of Dini's test, otherwise it would have its Fourier series converge. 
So, this function cannot be differentiable at $0$. 
Else, the function would be Lipschitz at $0$, and thus Dini's test would apply. 
In fact, there is a much weaker condition that allows us to apply Dini's test.

\begin{definition}
A function, $f: T \to \R$, is $\Lip_\al$ at $x_0$ if there is a $c \in \R$ such that for all $x \in T$, $\absval{f(x_0) - f(x)} < c\absval{x_0-x}^\al$. 
\end{definition}

This is a very weak condition that is still stronger than continuity; it is right to think of this as being ``just a little better than continuity.''
Indeed, a $Lip_\al$ function has $\absval{f(x_0) - f(x)}$ approach $0$ ``quickly enough'' that we can apply Dini's test.
This is a way to think of Dini's test; a continuous function that has $\absval{f(x_0) - f(x)}$ approach $0$ ``quickly enough'' has its Fourier series converge at $x_0$. 
Because being $Lip_\al$ is very weak, we should think of Dini's test as telling us that ``continuity is almost good enough.''


Another interesting point to make is that the set of functions we constructed is dense. 
This means that any function can be approximated by a function whose Fourier series diverges at $0$. 
What we can take away from this is that such functions aren't particularly rare: in a sense, these functions are always ``just around the block,'' rather than ``hiding away in dark corners.''
In other words, the result we had hoped for fails miserably. 
However, that is fine; this provides for a richer, more interesting theory of Fourier Analysis.

\end{document}