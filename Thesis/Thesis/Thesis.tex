%Comments go here.
%Remember, Max: No multiline comments here. LaTeX land isn't quite as good as it should be.
%Anyway, let's get on this:
%Doing numbering like this from now on:
%0.abc: a: math draft #. b: exposition draft # c: formatting draft number.

%Current status:
%Barebones:
%Parts 1-3: Edited, waiting approval.
%Parts 4-6: Proofs written. Old Man Approved!
%Baire's Theorem: Edited, awaiting approval.
%Main part: Written, Old Man Approved!.

%Exposition:
%Introduction: First draft done. Awaiting Feedback.
%Parts 1-3: Begun, edit when awake
%Parts 4-6: Begun, edit when awake
%Baire's Theorem: Begun, edit when awake
%Main part: Not begun
%Outro: Not begun

%Formatting:
%Will figure out later.


\documentclass{amsart}


\newcommand{\tab}{\hspace{4mm}} %Spacing aliases
\newcommand{\shunt}{\vspace{20mm}}

\newcommand{\sd}{\partial} %Squiggle d


\newcommand{\al}{\alpha} %Steal ALL of Dr. Kable's Aliases! MWAHAHAHAHA!
\newcommand{\be}{\beta}
\newcommand{\ga}{\gamma}
\newcommand{\Ga}{\Gamma}
\newcommand{\de}{\delta}
\newcommand{\De}{\Delta}
\newcommand{\ep}{\epsilon}
\newcommand{\vep}{\varepsilon}
\newcommand{\ze}{\zeta}
\newcommand{\et}{\eta}
\newcommand{\tha}{\theta}
\newcommand{\vtha}{\vartheta}
\newcommand{\Tha}{\Theta}
\newcommand{\io}{\iota}
\newcommand{\ka}{\kappa}
\newcommand{\la}{\lambda}
\newcommand{\La}{\Lambda}
\newcommand{\rh}{\rho}
\newcommand{\si}{\sigma}
\newcommand{\Si}{\Sigma}
\newcommand{\ta}{\tau}
\newcommand{\ups}{\upsilon}
\newcommand{\Ups}{\Upsilon}
\newcommand{\ph}{\phi}
\newcommand{\Ph}{\Phi}
\newcommand{\vph}{\varphi}
\newcommand{\vpi}{\varpi}
\newcommand{\ch}{\chi}
\newcommand{\ps}{\psi}
\newcommand{\Ps}{\Psi}
\newcommand{\om}{\omega}
\newcommand{\Om}{\Omega}

\newcommand{\bbA}{\mathbb{A}}
\newcommand{\A}{\mathbb{A}}
\newcommand{\bbB}{\mathbb{B}}
\newcommand{\bbC}{\mathbb{C}}
\newcommand{\C}{\mathbb{C}}
\newcommand{\bbD}{\mathbb{D}}
\newcommand{\bbE}{\mathbb{E}}
\newcommand{\bbF}{\mathbb{F}}
\newcommand{\bbG}{\mathbb{G}}
\newcommand{\G}{\mathbb{G}}
\newcommand{\bbH}{\mathbb{H}}
\newcommand{\HH}{\mathbb{H}}
\newcommand{\bbI}{\mathbb{I}}
\newcommand{\I}{\mathbb{I}}
\newcommand{\bbJ}{\mathbb{J}}
\newcommand{\bbK}{\mathbb{K}}
\newcommand{\bbL}{\mathbb{L}}
\newcommand{\bbM}{\mathbb{M}}
\newcommand{\bbN}{\mathbb{N}}
\newcommand{\N}{\mathbb{N}}
\newcommand{\bbO}{\mathbb{O}}
\newcommand{\bbP}{\mathbb{P}}
\newcommand{\PP}{\mathbb{P}}
\newcommand{\bbQ}{\mathbb{Q}}
\newcommand{\Q}{\mathbb{Q}}
\newcommand{\bbR}{\mathbb{R}}
\newcommand{\R}{\mathbb{R}}
\newcommand{\bbS}{\mathbb{S}}
\newcommand{\bbT}{\mathbb{T}}
\newcommand{\bbU}{\mathbb{U}}
\newcommand{\bbV}{\mathbb{V}}
\newcommand{\bbW}{\mathbb{W}}
\newcommand{\bbX}{\mathbb{X}}
\newcommand{\bbY}{\mathbb{Y}}
\newcommand{\bbZ}{\mathbb{Z}}
\newcommand{\Z}{\mathbb{Z}}

\newcommand{\scrA}{\mathcal{A}}
\newcommand{\scrB}{\mathcal{B}}
\newcommand{\scrC}{\mathcal{C}}
\newcommand{\scrD}{\mathcal{D}}
\newcommand{\scrE}{\mathcal{E}}
\newcommand{\scrF}{\mathcal{F}}
\newcommand{\scrG}{\mathcal{G}}
\newcommand{\scrH}{\mathcal{H}}
\newcommand{\scrI}{\mathcal{I}}
\newcommand{\scrJ}{\mathcal{J}}
\newcommand{\scrK}{\mathcal{K}}
\newcommand{\scrL}{\mathcal{L}}
\newcommand{\scrM}{\mathcal{M}}
\newcommand{\scrN}{\mathcal{N}}
\newcommand{\scrO}{\mathcal{O}}
\newcommand{\scrP}{\mathcal{P}}
\newcommand{\scrQ}{\mathcal{Q}}
\newcommand{\scrR}{\mathcal{R}}
\newcommand{\scrS}{\mathcal{S}}
\newcommand{\scrT}{\mathcal{T}}
\newcommand{\scrU}{\mathcal{U}}
\newcommand{\scrV}{\mathcal{V}}
\newcommand{\scrW}{\mathcal{W}}
\newcommand{\scrX}{\mathcal{X}}
\newcommand{\scrY}{\mathcal{Y}}
\newcommand{\scrZ}{\mathcal{Z}}

\newcommand{\diam}{\text{\rm diam}}

\newcommand{\colorcomment}[2]{\textcolor{#1}{#2}} %First of these leaves in comments. Second one kills them.
%\newcommand{\colorcomment}[2]{}

\newcommand{\absval}[1]{\lvert #1 \rvert}
\newcommand{\norm}[1]{\|#1\|}

\usepackage{fancyhdr}
\usepackage{amssymb}
\usepackage[usenames,dvipsnames]{color}
\usepackage{amsmath}
\usepackage{verbatim} 
\usepackage{slashed}
\usepackage[mathscr]{euscript}

\newtheorem{thm}{Theorem}[section]
\newtheorem{prop}[thm]{Proposition}
\newtheorem{lem}[thm]{Lemma}
\newtheorem{cor}[thm]{Corollary} 

\theoremstyle{definition}
\newtheorem{definition}[thm]{Definition}
\newtheorem{example}[thm]{Example}



\begin{document}

\title{The Existence of a Continuous Function Whose Fourier Series Diverges at a Point}
\author{Raymond Maxwell Jeter}

\begin{abstract}
There exists a continuous function whose Fourier series diverges at a point. 
\colorcomment{red}{In fact, the set of continuous functions whose Fourier series diverges at $0$ is dense in $C([0,2 \pi])$.}
\end{abstract}

\maketitle

\colorcomment{Thistle}{Note to Dr. Ullrich: I'm colorcoding this thing. Black is text I half-think should go in the final thing. Pink-Purple are direct statements to you. Red is placeholders. Blue is exposition, which I'll need lots of help with.}

\colorcomment{Thistle}{I need to make sure that I'm not wrong about the history here. Do we have a book I can reference or something?}

\colorcomment{Thistle}{Speaking of references, what should I do with respect to citations?}

\colorcomment{Thistle}{I also need you to explain what a ``Modulus of Continuity'' is. The version of Dini's test we were given in class totally skipped over that explanation and just gave us the theorem I state below.}

\section{Introduction}

%Exposition here goes like this:
%Say what Fourier believed.
%State Dini's test.
%Explain how Dini's test covers a lot of functions, and that it's only just a bit stronger than continuity.
%Explain how missing this gap makes sense given the tools Fourier had.
%It look like it do but it don't.

\colorcomment{RoyalBlue}{In \colorcomment{red}{paper} Fourier said \colorcomment{red}{quote} 
Now, observe Dini's Test;}

\begin{thm}
\colorcomment{red}{If $f \in L^1[0, 2\pi]$, $x_0 \in f^{-1}(\R)$, and $F(x) = \frac{f(x+x_0) - f(x_0)}{x}$ is integrable over some open neighborhood of $x_0$, then $f$'s Fourier series converges at $x_0$.}
\end{thm}

\colorcomment{RoyalBlue}{The condition required to apply Dini's test is ``Just a little better than continuity''; \colorcomment{red}{reasons}.
\colorcomment{red}{Some stuff.}}

\colorcomment{RoyalBlue}{However, as this paper will demonstrate, this is not true.}

\section{Notation}

\begin{definition}
If $X$ is a metric space, $r$ is a real number with $r>0$, and $x \in X$, then $B(x,r)$ is the $r$-ball centered at $x$.
\end{definition}

To reduce the complexity of notation, we say that

\begin{definition}
$T = [0, 2 \pi]$
\end{definition}

\begin{definition}
If $f$ is a continuous function, $n \in \N$, and $x \in \R$, then $S_n(f,x) = \sum\limits_{j=-n}^n c_n e^{-ijx}$, where $c_n = \frac{1}{2 \pi} \int\limits_0^{2\pi} f(x)e^{ijt} dx$. We call this the $n$-th partial sum of the Fourier series of $f$.
\end{definition}

\begin{definition}
If $f$ is a continuous function, then $\norm{f} = \sup\limits_{t \in T} \{\absval{f(t)}\}$
\end{definition}

\begin{definition}
For each $n \in \N$, we define $D_n: T \to \R$ by


\begin{displaymath}
   D_n(x) = \left\{
     \begin{array}{lr}
       1, &(n=0),\\
       \sum\limits_{j=0}^n e^{ijx}, &(n > 0).
     \end{array}
   \right.
\end{displaymath} 
\end{definition}

Also, we freely use the following well-known theorems:

\begin{thm}
\begin{displaymath}
   D_n(x) = \left\{
     \begin{array}{lr}
       \frac{\sin((n+\frac{1}{2})x)}{\sin(\frac{x}{2})}, &(x \neq 0),\\
       1+2n, &(x=0).
     \end{array}
   \right.
\end{displaymath}
\end{thm}

\begin{thm}
For any $f \in C(T)$, $(D_n \ast f)(x) = S_n(f,x)$. 
\end{thm}

\section{Continuous Functions With Some Partial Sum of its Fourier Series Larger Than A Given Real Number}

\colorcomment{Thistle}{I know you said that we don't need an outline of the proof first, but I think it's a good idea. Why do you think it's not a good idea?}

\colorcomment{RoyalBlue}{A rough sketch of the proof follows. 
First, for each $A \in \R$ we build a bounded, continuous function with some partial sum of its Fourier series larger than $A$ at $0$. 
We then consider sets of these functions with some partial sum of their Fourier series larger than some number at the point $0$.
We then show that these sets are open and dense. 
After that, we apply Baire's Theorem, which yields a dense, open set of functions whose Fourier series are unbounded at the point $0$. 
This dense, open set being nonempty yields a continuous function whose Fourier series diverges at a point.}

\subsection{The Construction of Bounded, Continuous Functions With Some Partial Sum of its Fourier Series Larger Than A Given Real Number}

\colorcomment{RoyalBlue}{Inspiration for constructing the titular functions comes from the following theorem.}

\begin{thm}
If for each $n \in \N$, we define $c_n \in \R$ to be

\begin{displaymath}
c_n = \frac{1}{2 \pi} \int\limits_0^{2\pi} \absval{D_n(x)} dx,
\end{displaymath}

then $c_n \to \infty$.
\end{thm}

\begin{proof}

First: $\absval{D_n(x)} \geq \frac{\frac{1}{2}}{\sin(\frac{x}{2})}$ if $x \in [\frac{\pi k}{6},\frac{5 \pi k}{6}]$ for some $k \in \N$.

This naturally yields the inequality $\int\limits_0^{2\pi} \absval{D_n(x)} dx \geq \sum\limits_{j=1}^n \frac{2 \pi}{6n \sin(\frac{5\pi j}{6(n+\frac{1}{2})})}$. 

\colorcomment{red}{*An illustration goes here*.}

Also, recall that for all $x \in T$, $x > \sin(x)$. 
So, $\frac{1}{x} < \frac{1}{\sin(x)}$ for all $x \in T$.

This means that we have

\begin{align*}
c_n &\geq \frac{1}{2 \pi} \sum\limits_{j=1}^n \frac{2 \pi}{6n \sin(\frac{5\pi j}{6(n+\frac{1}{2})})}\\
&= \frac{1}{6} \sum\limits_{j=1}^n \frac{1}{n \sin(\frac{5\pi j}{6(n+\frac{1}{2})})}\\
&\geq \frac{1}{6} \sum\limits_{j=1}^n \frac{1}{6n \frac{5\pi j}{6(n+\frac{1}{2})}}\\
&= \frac{1}{6} \sum\limits_{j=1}^n \frac{\frac{n+\frac{1}{2}}{n}}{5\pi j}\\
&\geq \frac{1}{6} \sum\limits_{j=1}^n \frac{1}{5\pi j}
\end{align*}

So $c_n \geq \frac{1}{30 \pi} \sum\limits_{j=1}^n \frac{1}{ j}$, and $\sum\limits_{j=1}^n \frac{1}{j} \to \infty$. So $c_n \to \infty$.

\end{proof}

\colorcomment{RoyalBlue}{It is important to determine how large a function's partial sum of its Fourier series can get at $0$. 
So we make the following definition:}

\begin{definition}
If $f \in C(T)$ , then $Mf = \sup\{\absval{S_m(f,0)} : m \in \N\}$ 
\end{definition}

\colorcomment{RoyalBlue}{So, we want to build a bounded function with $Mf >A$ for some $A \in \R$.
Constructing a sequence of functions, $g_n$, such that $Mg_n \to \infty$ would suffice.
We build this sequence rather naturally out of the proof of the above theorem.}

\begin{thm}
For every real number, $A$, there is an $f \in C(T)$ such that $\norm{f} = 1$ and $Mf > A$. 
\end{thm}

\begin{proof}
For each $n \in \N$, choose $g_n: T \to \R$ with $\norm{g_n} = 1$, with $g_nD_n(t) \geq 0$ for all $t \in T$,  $g_n(t) = 1$ if $t \in [\frac{\pi k}{6},\frac{5 \pi k}{6}]$ for some odd $k \in \N$  and $g_n(t) = -1$ if $t \in [\frac{\pi k}{6},\frac{5 \pi k}{6}]$ for some even $k \in \N$.

\colorcomment{Thistle}{An illustration goes here, maybe? If I decide to make pretty pretty pictures it might be worth it.}

So, similar to the above,

\begin{align*}
g_n \ast D_n(0) &\geq \frac{1}{2 \pi} \sum\limits_{j=1}^n \frac{2 \pi}{6n \sin(\frac{5\pi j}{6(n+\frac{1}{2})})}\\
&= \frac{1}{6} \sum\limits_{j=1}^n \frac{1}{n \sin(\frac{5\pi j}{6(n+\frac{1}{2})})}\\
&\geq \frac{1}{6} \sum\limits_{j=1}^n \frac{1}{6n \frac{5\pi j}{6(n+\frac{1}{2})}}\\
&= \frac{1}{6} \sum\limits_{j=1}^n \frac{\frac{n+\frac{1}{2}}{n}}{5\pi j}\\
&\geq \frac{1}{6} \sum\limits_{j=1}^n \frac{1}{5\pi j}
\end{align*}

So $g_n \ast D_n(0) \to \infty$. In other words, for each $A$ there is an $n \in \N$ such that $Mg_n > A$, which gives us what we wanted.

\end{proof}

\begin{thm}
For every $\ep >0$ and every real number, $A$, there is an $f \in C(T)$ such that $\norm{f} = \vep $ and $Mf > A$. 

\end{thm}

\begin{proof}
There is an $f \in C(T)$ such that $\norm{f} = 1$ and $Mf > \frac{A}{\vep}$.

Observe that $\vep f$ has $\norm{\vep f} = \vep$ and $M(\vep f) > A$.

\colorcomment{Thistle}{How should I make M (blah) not look bad? Should I do parenthesis stuff, or just leave off the parenthesis and hope the reader gets it?}
\end{proof}

\subsection{Building Open, Dense Sets of Continuous Functions With Some Partial Sum of its Fourier Series Larger Than A Given Real Number}

Next, we consider the following collection of sets: for each $n \in \N$, we define

\begin{displaymath}
E_n = \{f \in C(T) : Mf > n\}
\end{displaymath}

\begin{thm}
For all $n \in \N$, $E_n$ is dense.
\end{thm}

\begin{proof}

We show that given $\vep>0$ and $f \in C(T)$, $B(f,\vep) \cap E_n$ is nonempty. 

If $Mf > n$, then $f \in E_n$ and $f \in B(f, \vep)$, so $B(f, \vep) \cap E_n$ is nonempty.

Next, if $Mf \leq n$, there is a $g$ such that $Mg > 2n$ and $\norm{g} < \vep$. 
Now, $M(f+g) \geq n$, so $f+g \in E_n$. 
Also, $\norm{f+g -f} < \vep$, so that $f+g \in B(f, \vep)$. 
So $B(f, \vep) \cap E_n$ is nonempty.

\end{proof}

\begin{thm}
For all $n \in \N$, $E_n$ is open.
\end{thm}

\begin{proof}
Consider $\la_n: C(T) \to \R$ given by

\begin{displaymath}
\la_n(f) = S_n(f,0)
\end{displaymath}

Now, consider the following:

\begin{align*}
E_N &= \{f \in C(T): Mf> N\} \\
&= \bigcup\limits_{n \in \N} \{f \in C(T) : \la_n > N\}\\
&= \bigcup\limits_{n \in \N} \la^{-1}((N, \infty))
\end{align*}

It is sufficient to show that $\la_n$ is continuous for all $n \in \N$; if so, then the right hand side of the above equation is open.
We do this by showing that $\la_n$ is Lipschitz.

Let $f, g \in C(T)$. Then we have:

\begin{align*}
\absval{\la_n(f) - \la_n(g)} &= \absval{f \ast D_n(0) - g \ast D_n(0)} \\
&= \absval{(f-g) \ast D_n(0)} \\
&= \absval{\int\limits_0^{2\pi} (f(t)-g(t))D_n(t)dt} \\
&\leq \int\limits_0^{2\pi} \absval{(f(t)-g(t))D_n(t)}dt \\
&= \int\limits_0^{2\pi} \absval{(f(t)-g(t))}\absval{D_n(t)}dt \\
&\leq \int\limits_0^{2\pi} \norm{f-g}\absval{D_n(t)}dt \\
&= \norm{f-g} \int\limits_0^{2\pi} \absval{D_n(t)}dt
\end{align*}

Because $\int\limits_0^{2\pi} \absval{D_n(t)}dt$ is constant given $n$, this means that for any $f,g \in C(T)$, $\absval{\la_n(f) - \la_n(g)} \leq c_n \norm{f-g}$ for some constant $c_n$; $\la_n$ is Lipschitz.

\end{proof}

\begin{thm}
$C(T)$ is a complete metric space.
\end{thm}

\begin{proof}
We want to show that any Cauchy sequence of continuous functions, $(f_n)$, converges to a continuous function.

$\R$ is complete: so for each $x \in T$, $(f_n(x))$ converges to a point. We define $f: T \to \R$ by $f(x) = \lim\limits_{n \to \infty} f_n(x)$.

Now: let $\vep >0$. Choose $N$ such that for all $n,m \geq N$, ${f_n-f_m} < \frac{\vep}{2}$. 
Then for all $x \in T$ and $n,m \geq N$, , we have $\absval{f_n(x) - f_m(x)} < \frac{\vep}{2}$.
By taking a limit as $m \to \infty$, we have for all $x \in T$ and $n \geq N$, $\absval{f_n(x) - f(x)} \leq \frac{\vep}{2}$.

So $\norm{f_n - f} \leq \frac{\vep}{2} < \vep$ for all $n \geq N$; in other words, $f_n \to f$.

Also, $f$ is continuous because $f_n$ is a sequence of continuous functions that converges to $f$ uniformly.

\end{proof}

\colorcomment{RoyalBlue}{By the above three theorems, we have that $E_1, E_2 \ldots$ is a collection of open, dense sets in a complete metric space.}

\subsection{Baire's Theorem}

\colorcomment{RoyalBlue}{So, we now have a collection of open, dense sets in a complete metric space. 
We want to show that their intersection is nonempty; a dramatic way of doing this is to show that the intersection is dense. 
A flair for the dramatic is necessary in life and mathematics, so this is how we should proceed.}

In order to facilitate the proof of Baire's Theorem, we prove the following lemma:

\begin{lem}
If $X$ is a complete metric space, $C_1, C_2 \ldots$ are closed, nonempty subsets of $X$, $C_{n+1} \subset C_n$ for all $n \in \N$ and $\diam(C_n) \to 0$, then $\bigcap C_n$ has exactly one point.
\end{lem}

\begin{proof}
First we show that $\bigcap C_n$ is nonempty: 

Observe that $\bigcap\limits_{i=0}^n C_i$ is nonempty for all $n \in \N$. 
So there is a sequence $(x_n)$ such that $x_n \in \bigcap\limits_{i=0}^n C_i$ for all $n \in \N$. 
The sequence $(x_n)$ is Cauchy because $\diam(C_n) \to 0$, so $(x_n) \to x$ for some $x \in X$.
Now, $x \in \bigcap\limits_{i=0}^n C_i$ for all $n \in \N$.
So $x \in \bigcap C_n$: $\bigcap C_n \neq \emptyset$.

Next, we show that $\bigcap C_n$ has exactly one point:

Let $x, y \in \bigcap C_n$, with $x \neq y$. 
Then $d(x,y) = \vep$ for some $\vep >0$. 
However, $x,y \in C_n$ for all $n \in \N$,
so $\diam(C_n) \geq \vep$ for all $n \in \N$. 
So $\diam(C_n) \slashed{\to} 0$, contrary to our assumption.
\end{proof}

\begin{thm}If $X$ is a complete metric space, and $E_1, E_2, \ldots$ are dense, open subsets of $X$, then

\begin{displaymath}
\bigcap\limits_{n = 1}^{\infty} E_n
\end{displaymath}

is dense.
\end{thm}

\begin{proof}
We need to show that in a complete metric space, $X$, that given a collection of dense open sets, $E_1, E_2 \ldots$, and a nonempty open set $U$, that $\bigcap E_n \cap U$ is nonempty.

First, $E_1 \cap U$ is nonempty and open.
So there is a ball, $B(x_1,r_1) \subset E_1 \cap U$.
By taking $\vep_1 = \min(\frac{r_1}{2},1)$, we have $\overline{B(x_1,\vep_1)} \subset E_1 \cap U$.
Define $C_1 = \overline{B(x_1,\vep_1)}$.
Observe that $C_1$ is closed and nonempty. Also, $C_1 \subset U$.

Now, assume that for all $N \in \N$ such that $N < n$, we have constructed $C_{N}$, $x_{N}$, and $\vep_{N}$ such that $C_{N} = \overline{B(x_{N},\vep_{N})}$, $C_{N} \subset C_{N-1}$ (if $N >1$), $C_{N} \subset E_{N}$, and $\vep_{N} < \frac{1}{N}$.

Note that $E_n \cap B(x_{n-1},\vep_{n-1})$ is nonempty and open.
So there is a ball, $B(x_n,r_n) \subset E_n \cap B(x_{n-1},\vep_{n-1})$.
By taking $\vep_n = \min(\frac{r_n}{2},\frac{1}{n})$, we have $\overline{B(x_n,\vep_n)} \subset E_n \cap B(x_{n-1},\vep_{n-1})$.
Define $C_n = \overline{B(x_n,\vep_n)}$.
Observe that $C_n$ is closed and nonempty. Also, $C_n \subset C_{n-1}$.

So we have a collection of closed, nonempty, nested sets.
Also, note that because $\vep_n \to 0$, we have that $\diam(C_n) \to 0$. 
So $\bigcap\limits_{n=1}^\infty C_n$ has exactly one point.
This point is also in $U$, because $C_1 \subset U$.
This point is also in $E_n$ for all $n \in \N$, because $C_n \subset E_n$ for all $n \in \N$.
So $\bigcap\limits_{n=1}^\infty E_n \cap U$ isn't empty.

\end{proof}

\section{The Existence of a Continuous Function Whose Fourier Series Diverges at a Point}

\begin{thm}
There exists a continuous function whose Fourier series diverges at a point.
In fact, the set of functions in $C(T)$ whose Fourier series diverges at $0$ is dense.
\end{thm}

\begin{proof}

Recall that 

\begin{displaymath}
E_n = \{f \in C(T) : Mf > n\}
\end{displaymath}

is a collection of open, dense subsets of a complete space, $C(T)$. 
So Baire's Theorem applies: $\bigcap\limits_{n=1}^\infty E_n $ is dense.
The above set is precisely the set of functions, $f \in C(T)$, such that $Mf >n$ for all $n \in \N$.
In other words, this is a subset of the set of functions whose Fourier series diverges at $0$.
So the set of functions whose Fourier series diverges at $0$ contains a dense subset, it is dense.

Dense subsets of $C(T)$ are nonempty, so in fact there is a continuous function whose Fourier series diverges at a point.

\end{proof}

\section{Closing Remarks}

\colorcomment{RoyalBlue}{Consider the function we've shown exists: it rather obviously must fail the hypotheses of Dini's test, otherwise it would have its Fourier series converge. 
So, there's continuous functions such that \colorcomment{red}{Modulus of continuity is pretty bad}.
Now, \colorcomment{red}{Some examples go here.}}

\colorcomment{RoyalBlue}{Another interesting point to make is that the set of functions we constructed is dense. 
This means that any function can be approximated by a function whose Fourier series diverges at $0$. 
\colorcomment{red}{$\therefore$ lol.}}

\end{document}